\documentclass[12pt, letterpaper]{article}
\usepackage[utf8]{inputenc}
\usepackage[letterpaper, margin=0.5in]{geometry}
\usepackage{amsmath}
\usepackage{amssymb}
\usepackage{amsthm}


\title{STA 571 HW1}
\author{Ryan Tang}
\date{September 9th 2022}

\begin{document}
\maketitle

\section{Excerise 2.6}
\paragraph{(a)}
Let $H \in \{1, . . . , K\}$ be a discrete random variable, and let e1 and e2 be the observed values of two 
other random variables E1 and E2. Suppose we wish to calculate the vector
\[
    \vec{P}(H|e_1, e_2) = [P(H = 1|e_1, e_2), . . . , P (H = K|e_1, e_2)]^T
\]

Which of the following sets of numbers are sufficient for the calculation?
\begin{enumerate}
    \item $P(e_1, e_2), P(H), P(e_1|H), P(e_2|H)$
    \item $P(e_1, e_2), P(H), P(e_1, e_2|H)$
    \item $P(e_1|H), P(e_2|H), P(H)$
\end{enumerate}

\paragraph{Answer}
Without any more underlying assumptions, only point (2) is sufficient.

The conditional probability of $H$ can be expanded based off the Bayes rule.
\[
    P(H|e_1, e_2) = \frac{P(e_1, e_2 | H)P(H)} {P(e_1, e_2)}
\]


\paragraph{(b)}
Now suppose we now assume $E_1 \perp E_2|H$

\paragraph{Answer}
Now all 3 are sufficient.

For (1), given the conditional independence, $P(e_1, e_2|H) = P(e_1|H) P(e_2|H)$.

For (2),
\begin{align*}
    P(e_1, e_2) &= \int P(e_1|H)P(e_2|H)P(H) \, dH \\
                &= \int P(e_1, e_2|H)P(H) \, dH \\
                &= \int P(e_1, e_2, H) \, dH \\
                &= P(e_1, e_2)
\end{align*}

\section{Excerise 2.7}
Pairwise independence does not imply mutual independence

\paragraph{Answer}
Suppose a simple example $X_1, X_2, X_3 \in {0, 1}$ three random binary variables.

Mutual independence $\rightarrow P(X_1, X_2, X_3) = P(X_1) P(X_2) P(X_3)$.

Pairwise independence $\rightarrow P(X_1, X_2, X_3) = P(X_1|X_2, X_3) P(X_2) P(X_3)$ instead.

If we suppose \textit{mutual independence} $\rightarrow$ \textit{pairwise independence}, then we have
\[
    P(X_1) = P(X_1 | X_2, X_3)
\]
However, the statement is violated given the following example.
Say $X_2$ and $X_3$ are two fair binary random variables $\in \{0, 1\}$ and $X_1$ is, instead,
generated in the following way where $P(X_1 = 0 | X_2 = X_3) = 1$ and vice versa. 
\begin{center}
\begin{tabular}{|c c c|} 
    \hline
    X2 & X3 & X1 \\ [0.5ex] 
    \hline\hline
    0 & 0 & 0 \\ 
    \hline
    1 & 1 & 0 \\
    \hline
    1 & 0 & 1 \\
    \hline
    0 & 1 & 1 \\
    \hline
\end{tabular}
\end{center}

\section{Excerise 2.8}
Conditional independence iff joint factorizes $X \perp Y |Z$ iff $p(x, y|z) = p(x|z)p(y|z)$

Proofs $X \perp Y |Z$ iff $p(x, y|z) = g(x, z)h(y, z)$

\paragraph{Answer}
Let's assume the above statement is true, then we can write
\begin{proof}
\begin{align*}
    \iint p(x, y|z) \, dxdy &= \iint g(x, z)h(y, z) \, dxdy \\
    p(x|z) &= \int_Y g(x, z)h(y, z) \, dy \propto g(x, z) \\
    p(y|z) &\propto h(y, z) \\
    \therefore p(x|z)p(y|z) &= c(z)g(x, z)h(y, z)
\end{align*}

Because both side are proper probability and the intergral of the entire space should sum up to 1.
Hence, $c(z)$, the constant has to be 1 which implies
\[
    g(x, z)h(y, z) \Rightarrow p(x|z)p(y|z)
\]
\end{proof}

\section{Excerise 2.12}
\begin{proof}
\begin{align*}
    \mathbb{MI}(X, Y) &= KL[p(x, y) || p(x)p(y)] \\
        &= \sum_X \sum_Y p(x, y) \log \frac{p(x, y)}{p(x)p(y)} \\
        &= \sum_X \sum_Y p(x, y) \log \frac{p(x, y)}{p(x)} - \sum_X \sum_Y p(x,y) \log p(y) \\
        &= \sum_X p(x) \sum_Y p(y|x) \log p(y|x) - \sum_Y p(y) \log p(y) \sum_X p(x|y) \\
        &= - \sum_X p(x) H(Y|X=x) - \sum_Y p(y) \log p(y) \\
        &= H(Y) - H(Y|X) \\
        &= H(X) - H(X|Y)
\end{align*}
\end{proof}

\section{Excerise 2.15}
\begin{proof}
\begin{align*}
    \hat{\theta} &= \arg \max_{\theta} q(X|\theta) && \text{$X \in R^{N \times D}$} \\
      &= \arg \max_{\theta} \prod_{i\in X} q(x_i|\theta) \\
      &= \arg \max_{\theta} \frac{1}{N} \sum_{i\in X} \log q(x_i|\theta) \\
      &= \arg \min_{\theta} \mathop{\mathbb{E}}_X [- \log q(x|\theta)]  && \text{MLE result} \\ \\
\end{align*}

\begin{align*}
    KL(p || q(X|\theta)) &= \sum_X p(x) \log \frac{p(x)}{q(x|\theta)} \\
      &= \mathop{\mathbb{E}}_X [\log p(x) - \log q(x|\theta)] \\
      &= \arg \min_{\theta} \mathop{\mathbb{E}}_X [- \log q(x|\theta)] && \text{remove constant} \\
      &= \text{MLE $\hat{\theta}$}
\end{align*}
\end{proof}

\section{Excerise 3.20} 
\paragraph{(a)}
For joint distribution without factorization on conditional independence, a look-up table is the
best approach to represent the joint distribution. In our binary random variable case with
$C$ classes and $D$ features, the look-up table has the size of $[C \times (2^D - 1)]$ or $O(C2^D)$.

\paragraph{(b)}
The Naive Bayes should perform better if we do not have much data.

\paragraph{(c)}
If we have infinite sample data, the full-distribution model should perform better.

\paragraph{(d)}
Both versions should have $O(ND)$. It needs at least one sweep of all available elements in the data set.

\paragraph{(e)}
Complexity on a single test sample takes $O(CD)$ for Naive Bayes. It needs to go through all the features
for each class. Full Bayes takes $O(C)$; it only performs look-up for all classes.

\paragraph{(f)}
Missing feature data in the test set should not affect the performance in Naive Bayes because features are
conditionally independent of each other given a class. Hence $O(C \cdot (v+h))$.

On the other hand, Full Bayes needs to impute the average for these missing averages which takes
exponential time during a sweep of the look-up table. Hence $O(v \cdot 2^h)$.

\section{Excerise 3.22}

\begin{center}
\begin{tabular}{l|lllllll}
\multicolumn{1}{c|}{Label}  & \multicolumn{1}{c}{Spam} & \multicolumn{1}{c}{Spam} & \multicolumn{1}{c}{Spam} & Ham & Ham & Ham & Ham \\ \hline
\multicolumn{1}{c|}{secret} & \multicolumn{1}{c}{}     & \multicolumn{1}{c}{1}    & \multicolumn{1}{c}{1}    &     & 1   &     &     \\
\multicolumn{1}{c|}{offer} & \multicolumn{1}{c}{1} & \multicolumn{1}{c}{1} & \multicolumn{1}{c}{} &   &   &   &   \\
\multicolumn{1}{c|}{low}   & \multicolumn{1}{c}{}  & \multicolumn{1}{c}{}  & \multicolumn{1}{c}{} & 1 &   &   & 1 \\
\multicolumn{1}{c|}{price} & \multicolumn{1}{c}{}  & \multicolumn{1}{c}{}  & \multicolumn{1}{c}{} & 1 &   &   & 1 \\
valued                     &                       &                       &                      & 1 &   &   &   \\
customer                   &                       &                       &                      & 1 &   &   &   \\
today                      &                       & 1                     &                      &   & 1 &   &   \\
dollar                     & 1                     &                       &                      &   &   &   &   \\
million                    & 1                     &                       &                      &   &   &   &   \\
sports                     &                       &                       &                      &   & 1 & 1 &   \\
is                         &                       &                       & 1                    &   &   & 1 &   \\
for                        &                       &                       &                      & 1 &   &   &   \\
play                       &                       &                       &                      &   & 1 &   &   \\
healthy                    &                       &                       &                      &   &   & 1 &   \\
pizza                      &                       &                       &                      &   &   &   & 1
\end{tabular}
\end{center}

\[ \theta_{spam} = \frac{3}{7} \]
\[ \theta_{secret|spam} = \frac{2}{3}, \theta_{dollar|spam} = \frac{1}{3} \]
\[ \theta_{secret|ham} = \frac{1}{4}, \theta_{sports|ham} = \frac{1}{2} \]

\section{Excerise 4.21}
\paragraph{(a)} 
We are given the following: $\mu_1 = 0$, $\sigma_1^2 = 1$, $\mu_2 = 1$, $\sigma_2^2 = 10^6$.
The decision boundary of QDA between two classes are given by
\begin{align*}
    p(y=1 | x, \theta_1) &= p(y=2 | x, \theta_2) \\
    p(x|y=1 | x, \theta_1) p(y=1|\pi_1) &= p(y=2|x,\theta_2)p(y=2|\pi_2) \\
    \log p(x|y=1 | x, \theta_1) &= \log p(y=2|x,\theta_2) && \text{Given $\pi_1 = \pi_2 = 0.5$} \\
    -\log \sigma_1 - \frac{1}{2}(\frac{x-\mu_1}{\sigma_1})^2 &= -\log \sigma_2 - \frac{1}{2}(\frac{x-\mu_2}{\sigma_2})^2
\end{align*}

After plugging in the given $\mu$ and $\sigma$ for both classes, the bound arrives at $-3.72 \le x \le 3.72$. It is a closed
bound that anything in between is class 1 and everything outside is class 2.

\paragraph{(b)}
Now $\sigma_2^2 = 1$ instead of $10^6$ this time. The bound is given by $x = 0.5$. Anything to the left is class 1 and vice versa.
It makes sense because $x = 0.5$ is the average of both means.

\section{Excerise 4.22}
We first start with the posterior and its likelihood proportionality.
\begin{align*}
    p(y=c|x,\theta_c) \propto p(x|y=c,\theta_c)p(y_c|\pi)
\end{align*}
Given equal prior to all 3 classes, $p(y=1) = p(y=2) = p(y=3) = 1/3$. The classification task simplifies to
whichever class has the highest multivariate normal density of observing the given data, $p(x|y=c,\theta_c)$,
with respect to its parameters. \\
(a) Softmax $= [0.35, 0.31, 0.34] \rightarrow $ Class 1 \\
(b) Softmax $= [0.347, 0.348, 0.305] \rightarrow $ Class 2, although really close

\section{Excerise 4.23}
We are given this data
\begin{center}
\begin{tabular}{ccc}
index & x  & label \\
1     & 67 & m     \\
2     & 79 & m     \\
3     & 71 & m     \\
4     & 68 & f     \\
5     & 67 & f     \\
6     & 60 & f    
\end{tabular}
\end{center}

\paragraph{(a)} Fitting a Naive Bayes using MLE on a one-dimensional feature and a normal generative model,
the resulting parameters are just the sample mean and variances. Hence,
\[
    \mu_m = 72.33, \sigma_m = 6.11, \pi_m = 0.5
\]
\[
    \mu_f = 65.00, \sigma_m = 4.36, \pi_m = 0.5
\]
with an equal likelihood prior. The general population has males and females distributed relatively evenly
as well as the samples.

\paragraph{(b)} To make a prediction on $p(y=m|x, \hat{\theta_m})$ with $x = 72$, we use the following equations.
\begin{align*}
    p(y=m|x, \hat{\theta_m}) &= \frac{p(x|y=m, \hat{\theta_m})\pi_m}{p(x|y=m, \hat{\theta_m})\pi_m + p(x|y=f, \hat{\theta_f})\pi_f}
\end{align*}
The resulting probability is 0.721.

\paragraph{(c)} Just use the QDA with a multivariate-normal generative model. So the likelihood becomes
\[
    p(\mathbf{x}|y=c, \mathbf{\theta}) = \mathcal{N}(\mathbf{x; \mu_x}, \mathbf{\Sigma_x})
\]
Instead of estimating the variance, we can use the covariances matrix between two or more variables.

\end{document}